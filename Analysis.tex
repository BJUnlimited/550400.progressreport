\documentclass[oneside,12pt]{report}  

% the dimensions of the page
\textheight=9.25in \topmargin=-0.5in   %See note in Chapter 8 of Sample Report about "Page scaling" option in Adobe
\textwidth=6.0in
\oddsidemargin=0.3in
\evensidemargin=0.3in  % Needed to balance even and odd pages in twoside print copy


% Useful packages
\usepackage{dtklogos}
\usepackage{amsmath}
\usepackage{bm}
%\usepackage[colorlinks=true,pagebackref,linkcolor=blue]{hyperref}
\usepackage{amsfonts}
\usepackage{amsthm}
\usepackage{amsmath}
\usepackage{algorithm}
\usepackage{algorithmic}
\usepackage{graphicx, subfigure}
\usepackage{caption}
\usepackage{excludeonly}
\usepackage{mathtools}
\usepackage{graphicx} 

%\usepackage{doc}
%% Following sets up logic and formatting for conditional twoside copying
%\usepackage{ifthen, color, fancyvrb}
%\usepackage{nextpage}\pagestyle{plain}
%\newcommand\myclearpage{\cleartooddpage
%  [\thispagestyle{empty}]
%  }

\DeclareMathOperator*{\argmin}{arg\ min}
\DeclareMathOperator*{\sign}{sign}

% Note special alternative codes for using TWO bibliographies; see cautionary note in
\DeclareGraphicsExtensions{ps,eps,PNG,png}

% Theorem-like command definitions:
\newtheorem{theorem}{Theorem}[chapter]
\newtheorem{lemma}{Lemma}[chapter]
\newtheorem{definition}{Definition}  % Note, this italicizes everything

% Print the chapter and sections in the toc
\setcounter{tocdepth}{1}

% Specify which files to typeset for this run (note that overall pagination is preserved)
%\includeonly{chapter1, chapter2}
% Specify which files NOT to typeset for this run (note that overall pagination is preserved)
%\excludeonly{}

% Groundwork for allowing double-sided copying with blank versos
\def\prefacesection#1{
\chapter*{#1}
\addcontentsline{toc}{chapter}{#1}
}

\begin{document}


\def\thefootnote{\fnsymbol{footnote}}

\thispagestyle{empty}

% The numbers below controls the amount of space between the following sections
\def\shiftdowna{0.32in}  % Adjust for balance
\def\shiftdownb{0.22in}  % Adjust for balance

% Set up the boiler plate at the top of the page

\begin{center}
\textbf{{\large Mathematical Modeling and Consulting }}\\

\vspace \shiftdowna
\includegraphics[width=0.5\textwidth]{jhu.png}\\

% Home Department
\vspace \shiftdowna
\underline {Sponsor}\\ 
\vspace{5pt}
\textbf{\large Sponsor Name} \\
\vspace\shiftdowna
\textbf{{Final Report}}

% TITLE
\vspace \shiftdowna
\textbf{{\Large Insurance Redlining}}

% STUDENTS
\vspace{0.35in}
\underline {Team Members}\\
\vspace{5pt}
John Doe (Project Manager), Home Department\\
\texttt{john.doe@jhu.edu} \\
\vspace{10pt}
Jane Doe (Report Coordinator), Home Department
%\texttt{jane.doe@jhu.edu}

% INSTRUCTOR
\vspace \shiftdownb
\underline {Academic Mentor} \\
\vspace{5pt}
\text{Dr.~N.~.H.~Lee}, Applied Mathematics and Statistics\\
\texttt{nhlee@jhu.edu}

% Consultants
\vspace \shiftdownb 
\underline {Consultant}\\
\vspace{5pt}
Jason Bourne\\

% DATE
\vspace \shiftdowna
Date: Last Complied on \today

\end{center}

\vfill  %Fill page to force following note to bottom
\footnoterule
\noindent \small{This project was supported by XYZ.}

% Begin ABSTRACT
\ifthenelse{\boolean{@twoside}}{\myclearpage}{}
\prefacesection{Abstract}

% Begin ACKNOWLEDGMENTS
\ifthenelse{\boolean{@twoside}}{\myclearpage}{}
\prefacesection{Acknowledgments}

% Table of contents, List of Figures, and List of Tables.
\ifthenelse{\boolean{@twoside}}{\myclearpage}{}
\tableofcontents

\ifthenelse{\boolean{@twoside}}{\myclearpage}{}
\listoffigures

\ifthenelse{\boolean{@twoside}}{\myclearpage}{}
\listoftables


\renewcommand{\thefootnote}{\arabic{footnote}}
\setcounter{footnote}{0}

\ifthenelse{\boolean{@twoside}}{\myclearpage}{}
%\include{A_Introduction}
%\include{B_TechnicalBackground}
%\include{C_ProblemStatement}
\include{D_Analysis}
\ifthenelse{\boolean{@twoside}}{\myclearpage}{}
\prefacesection{Analysis}
Because this model is based on qualitative observation it is imperative to clearly define the axioms and observations we have considered.
\paragraph{Observations/Axioms}
    \begin{itemize}
	\item Willingness of an individual member of a crowd to cheer depends on the number of people cheering around the individual,
	\item The speaker, performer, artist or other stimulus plays a role in cheering, in this case performance of the team,
	\item Defined intervals of time as rounds and as the number of rounds increases the number of people that cheer increase,
	\item Cheering of the individual depends on the internal state of the individual, i.e happy, sad, etc,
	\item There are different intensities and levels of cheering, 
	\item  These behavior levels or intensities behave as a continuous gradient.
		\end{itemize}

\paragraph {Simplifications and Assumptions}

The stimulus in this case would be the performance of the team. For the purpose of simplicity we will assume that the stimulus is average. From there, we can take a snap shot of the crowd's behavior and ignore the dependence of cheering on the team's performance. Each snap shot can be represented by a "round".  
\newline
For the internal state of the individual we can assign base "cheer number" using a stochastic process. We will simplify cheering as a probabilistic switch with a universal threshold that if exceeded the individual will be considered cheering otherwise the indivdual will not be cheering. This will simplify the problem because it will allow us to ignore the continuos gradient and different levels of cheering. 
\newline
We will simplify the relationship of the willingness to cheer and the surrounding individuals to only include the individuals that are directly surrounding the intended member.

\paragraph {Computational Simulation}

Using MATLAB, we start by generating an arbitrary sized $n~x~m$ matrix. The parameter can be changed based on the user's liking. The generated matrix represents the crowd where each element represents an indicidual in the crowd. The first matrix generated was filled randomly with  base "cheer numberl". Each element is a normal random variable with the mean set to 10 and the standard deviation of 1. We can use a normal random variable becasue we can consider cheering to be an introduced innovation \cite{DI2003}.  In general, the mean and standard deviation is intrinsic to each crowd. However, the reason the mean and standard deviation are assigned these values to ensure that it is highly unlikely that the base value is a negative number ( you need approximately more that 10 standard deviations below the mean to reach negative values). This makes the math much easier when considering the dependence on the surrounding members, as it will be revealed later. This can be expressed with the following equation:
\[ Let~ X_{ij}\sim Norm(10,1)\]

$i$={1,...,$n$}~and~$j$={1,...,$m$}\newline
 

Next, we set threshold to be one standard deviation above the mean, which is about 11. We then used  a boolean argument and go through the matrix to check which individual is above or equal to the threshold. If the individual or matrix element is at least at threshold a new matrix and the corresponding element is assigned a value of one, otherwise the element will be assigned a null value or zero. This can be represented with the following equation:
\[  Let~ Y_{ij}~ be~the~ element~ of~the~new~boolean~ matrix \]
\newline
then,
\[ Y_{ij}= \theta(X_{ij}\geq 11) ,~where~\theta(...) =1~if~argument~is~TRUE~and~=0~if~argument~is~FALSE\]
 \newline
The following step is to simulate the model and its dependence on the number of people cheering around the individual and the number of rounds. We define a round to be an arbitrary time interval (approximately 3-5 seconds, in this case). We then take snapshots of the crowds by running rounds. We update the base cheer number for each individual based on the following general qualitative formula:
\newline
\newline
Let ~$ S$~define~the~number~of~surrounding~cheering~fans,\newline
Let ~$ R$~define~the~round~number,\newline
Let ~$X'$~define~the~updated~cheering~number,\newline
Let ~$Y'$~define~the~updated~boolean~element,\newline
 \[ X'_{ij}= X_{ij}*S+R \]
  \[
Y'_{ij}=\theta(X'_{ij}\geq11)
\]
\newline
\newline 
The final output is a graphical representation of of the cheering behavior over a $R$ rounds.


%\include{E_Results}
%\include{F_Conclusion}

%\include{chapter1}
%\include{chapter2}
%\include{chapter3}
%\include{chapter4}
%\include{chapter5}
%\include{chapter6}


\appendix
\ifthenelse{\boolean{@twoside}}{\myclearpage}{}

\chapter{Lemmas}\label{Lemma}

\chapter{Glossary}\label{Glossary}
\ifthenelse{\boolean{@twoside}}{\myclearpage}{}
\prefacesection{Glossary}
\vspace{12pt} 

\vspace{8pt}
\noindent {\bf Cheer starters}. 
the equatorial plane in a northerly direction. 

\vspace{8pt}
\noindent {\bf Cheer number}. A frame of reference whose origin is the center of the earth and which does not rotate with respect to inertial space.

\vspace{8pt}
\noindent {\bf Earth-centered rotating frame}. A frame of reference whose origin is the center of the earth but which rotates with the earth. 

\vspace{8pt} \noindent {\bf Footprint}. The intersection of a visibility cone with the surface of the earth.

\vspace{8pt} \noindent {\bf Great circle of arc}. The shortest path between two points on the surface of the earth. 

\vspace{8pt} \noindent {\bf Groundtrack}.The location of the center of a visibility cone footprint on the surface of the earth.

\vspace{8pt}
\noindent {\bf Inclination}.  The angle between the normal to the orbit plane
and the normal to the equatorial plane.

\vspace{8pt} \noindent {\bf LEO}. An orbit with an altitude approximately below 2,000 km.

\vspace{8pt} \noindent {\bf Molniya orbit}. A highly elliptical orbit with an orbital period of half a day.

\vspace{8pt} \noindent {\bf Projection distance}. The distance between the center of the visibility cone footprint and a point of interest projected onto the plane orthogonal to the vector defining the visibility cone center and tangent to the earth surface.

\vspace{8pt}
\noindent {\bf Right ascension of the ascending node}. The angle
between the unit vector $\bm{X}$ and the point where the satellite crosses the
ascending node, measured counterclockwise when viewed from the north side of
the equatorial plane.


\ifthenelse{\boolean{@twoside}}{\myclearpage}{}
\chapter{Abbreviations}\label{Abbreviations}


\noindent ECI.  Earth-centered inertial frame

\vspace{5pt}

\noindent ECR.  Earth-centered rotating frame

\vspace{5pt}

\noindent LEO.  Low Earth Orbit  

\vspace{5pt}

\noindent RAAN. Right ascension of the ascending node

\vspace{5pt}

%\endinput

% Add your bibliography to Contents
\ifthenelse{\boolean{@twoside}}{\myclearpage}{\newpage}
\addtocontents {toc}{\protect \contentsline {chapter}{REFERENCES}{}}
\addcontentsline{toc}{chapter}{Selected Bibliography Including Cited Works}  

% Bibliography must come last.
\bibliographystyle{plain}
\renewcommand\bibname{Selected Bibliography Including Cited Works}
\nocite{*}  % List ALL references in your references, not just the ones cited in the text.
% This scheme automatically alphabetizes the Bibliography.
\bibliography{Biblio}
\end{document}
