\documentclass[oneside,12pt]{report}  

% the dimensions of the page
\textheight=9.25in \topmargin=-0.5in   %See note in Chapter 8 of Sample Report about "Page scaling" option in Adobe
\textwidth=6.0in
\oddsidemargin=0.3in
\evensidemargin=0.3in  % Needed to balance even and odd pages in twoside print copy


% Useful packages
\usepackage{dtklogos}
\usepackage{amsmath}
\usepackage{bm}
%\usepackage[colorlinks=true,pagebackref,linkcolor=blue]{hyperref}
\usepackage{amsfonts}
\usepackage{amsthm}
\usepackage{amsmath}
\usepackage{algorithm}
\usepackage{algorithmic}
\usepackage{graphicx, subfigure}
\usepackage{caption}
\usepackage{excludeonly}

\usepackage{graphicx} 

%\usepackage{doc}
%% Following sets up logic and formatting for conditional twoside copying
%\usepackage{ifthen, color, fancyvrb}
%\usepackage{nextpage}\pagestyle{plain}
%\newcommand\myclearpage{\cleartooddpage
%  [\thispagestyle{empty}]
%  }

\DeclareMathOperator*{\argmin}{arg\ min}
\DeclareMathOperator*{\sign}{sign}

% Note special alternative codes for using TWO bibliographies; see cautionary note in
\DeclareGraphicsExtensions{ps,eps,PNG,png}

% Theorem-like command definitions:
\newtheorem{theorem}{Theorem}[chapter]
\newtheorem{lemma}{Lemma}[chapter]
\newtheorem{definition}{Definition}  % Note, this italicizes everything

% Print the chapter and sections in the toc
\setcounter{tocdepth}{1}

% Specify which files to typeset for this run (note that overall pagination is preserved)
%\includeonly{chapter1, chapter2}
% Specify which files NOT to typeset for this run (note that overall pagination is preserved)
%\excludeonly{}

% Groundwork for allowing double-sided copying with blank versos
\def\prefacesection#1{
\chapter*{#1}
\addcontentsline{toc}{chapter}{#1}
}

\begin{document}


\def\thefootnote{\fnsymbol{footnote}}

\thispagestyle{empty}

% The numbers below controls the amount of space between the following sections
\def\shiftdowna{0.32in}  % Adjust for balance
\def\shiftdownb{0.22in}  % Adjust for balance

% Set up the boiler plate at the top of the page

\begin{center}
\textbf{{\large Mathematical Modeling and Consulting }}\\

\vspace \shiftdowna
\includegraphics[width=0.5\textwidth]{jhu.png}\\

% Home Department
\vspace \shiftdowna
\underline {Sponsor}\\ 
\vspace{5pt}
\textbf{\large Sponsor Name} \\
\vspace\shiftdowna
\textbf{{Final Report}}

% TITLE
\vspace \shiftdowna
\textbf{{\Large Insurance Redlining}}

% STUDENTS
\vspace{0.35in}
\underline {Team Members}\\
\vspace{5pt}
John Doe (Project Manager), Home Department\\
\texttt{john.doe@jhu.edu} \\
\vspace{10pt}
Jane Doe (Report Coordinator), Home Department
%\texttt{jane.doe@jhu.edu}

% INSTRUCTOR
\vspace \shiftdownb
\underline {Academic Mentor} \\
\vspace{5pt}
\text{Dr.~N.~.H.~Lee}, Applied Mathematics and Statistics\\
\texttt{nhlee@jhu.edu}

% Consultants
\vspace \shiftdownb 
\underline {Consultant}\\
\vspace{5pt}
Jason Bourne\\

% DATE
\vspace \shiftdowna
Date: Last Complied on \today

\end{center}

\vfill  %Fill page to force following note to bottom
\footnoterule
\noindent \small{This project was supported by XYZ.}

% Begin ABSTRACT
\ifthenelse{\boolean{@twoside}}{\myclearpage}{}
\prefacesection{Abstract}

% Begin ACKNOWLEDGMENTS
\ifthenelse{\boolean{@twoside}}{\myclearpage}{}
\prefacesection{Acknowledgments}

% Table of contents, List of Figures, and List of Tables.
\ifthenelse{\boolean{@twoside}}{\myclearpage}{}
\tableofcontents

\ifthenelse{\boolean{@twoside}}{\myclearpage}{}
\listoffigures

\ifthenelse{\boolean{@twoside}}{\myclearpage}{}
\listoftables


\renewcommand{\thefootnote}{\arabic{footnote}}
\setcounter{footnote}{0}

\ifthenelse{\boolean{@twoside}}{\myclearpage}{}
\prefacesection{Introduction}

\ifthenelse{\boolean{@twoside}}{\myclearpage}{}
\prefacesection{Results}

\begin{figure}[h]
    \begin{center}
        \includegraphics[width=\textwidth]{sample_graph.jpg}
    \end{center}
    \caption{Graphical representation of cheering in a 20 x 100 crowd for ten rounds.}
\end{figure}

\begin{figure}[h]
    \begin{center}
        \includegraphics[width=\textwidth]{sample_graph2.jpg}
    \end{center}
    \caption{Increasing overall percentage of people cheering for ten rounds.}
\end{figure}

\ifthenelse{\boolean{@twoside}}{\myclearpage}{}
\prefacesection{Remaining Work to be Done}
\paragraph{}
For our current model, we picked sample parameters for purposes of demonstration so far. For example, our initial threshold is 11 (one standard deviation above the normal distribution centered at 10), and the second threshold is 45. The simulation runs for 10 rounds for a crowd of 2,000 people (20 x 100). However, we have not yet determined an absolute set of model parameters for the entire project. For the remaining time we have to work on this project, we must pick a definitive initial threshold, the second threshold, the size of the crowd (number of rows and columns of the first matrix), the number of rounds, and the number of cheer starters.
\paragraph{}
We must also investigate the effect of cheer starters on the percentage of fans cheering in the final round. The current model does not take into account of cheer starters; rather, it simulates a stochastic, general crowd susceptible to the influence of people cheering around them. We could include an initial number of people already cheering and establish the effect on the percentage of people cheering in the final round. Logically, this would significantly increase the percentage of people cheering in the final round.
\paragraph{}
If time permits, we could also manually dictate where the cheer starters are in order to maximize the percentage cheering. We plan to add the cheer starters and place them randomly in the matrix but have no plans to place them in specific locations in the crowd. We can hypothesize that certain regions of the crowd will have more people cheering than other locations and certain patterns will produce a large percentage of overall people cheering than other patterns.
\paragraph{}
Finally, we can complete the project by polishing the code in MATLAB and writing R documentation for an R package. We will also need to write a final report that introduces the project, explains the approach, and analyzes the results. We will also prepare a final presentation for this project using beamer slides.

%\include{chapter1}
%\include{chapter2}
%\include{chapter3}
%\include{chapter4}
%\include{chapter5}
%\include{chapter6}


\appendix
\ifthenelse{\boolean{@twoside}}{\myclearpage}{}

\chapter{Lemmas}\label{Lemma}

\chapter{Glossary}\label{Glossary}

\vspace{12pt} 

\vspace{8pt}
\noindent {\bf Ascending node}. The point where the satellite crosses through
the equatorial plane in a northerly direction. 

\vspace{8pt}
\noindent {\bf Earth-centered inertial frame}. A frame of reference whose origin is the center of the earth and which does not rotate with respect to inertial space.

\vspace{8pt}
\noindent {\bf Earth-centered rotating frame}. A frame of reference whose origin is the center of the earth but which rotates with the earth. 

\vspace{8pt} \noindent {\bf Footprint}. The intersection of a visibility cone with the surface of the earth.

\vspace{8pt} \noindent {\bf Great circle of arc}. The shortest path between two points on the surface of the earth. 

\vspace{8pt} \noindent {\bf Groundtrack}.The location of the center of a visibility cone footprint on the surface of the earth.

\vspace{8pt}
\noindent {\bf Inclination}.  The angle between the normal to the orbit plane
and the normal to the equatorial plane.

\vspace{8pt} \noindent {\bf LEO}. An orbit with an altitude approximately below 2,000 km.

\vspace{8pt} \noindent {\bf Molniya orbit}. A highly elliptical orbit with an orbital period of half a day.

\vspace{8pt} \noindent {\bf Projection distance}. The distance between the center of the visibility cone footprint and a point of interest projected onto the plane orthogonal to the vector defining the visibility cone center and tangent to the earth surface.

\vspace{8pt}
\noindent {\bf Right ascension of the ascending node}. The angle
between the unit vector $\bm{X}$ and the point where the satellite crosses the
ascending node, measured counterclockwise when viewed from the north side of
the equatorial plane.


\ifthenelse{\boolean{@twoside}}{\myclearpage}{}
\chapter{Abbreviations}\label{Abbreviations}


\noindent ECI.  Earth-centered inertial frame

\vspace{5pt}

\noindent ECR.  Earth-centered rotating frame

\vspace{5pt}

\noindent LEO.  Low Earth Orbit  

\vspace{5pt}

\noindent RAAN. Right ascension of the ascending node

\vspace{5pt}

%\endinput

% Add your bibliography to Contents
\ifthenelse{\boolean{@twoside}}{\myclearpage}{\newpage}
\addtocontents {toc}{\protect \contentsline {chapter}{REFERENCES}{}}
\addcontentsline{toc}{chapter}{Selected Bibliography Including Cited Works}  

% Bibliography must come last.
\bibliographystyle{plain}
\renewcommand\bibname{Selected Bibliography Including Cited Works}
\nocite{*}  % List ALL references in your references, not just the ones cited in the text.
% This scheme automatically alphabetizes the Bibliography.
\bibliography{Biblio}
\end{document}
